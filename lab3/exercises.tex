% !Tex program = pdflatex
\documentclass[a4paper, leqno]{article}
 \usepackage[T2A]{fontenc}
\usepackage[utf8x]{inputenc} % Включаем поддержку UTF8  
\usepackage[russian]{babel}
\usepackage{amsmath}
\usepackage{bm}
\begin{document}
Начнем с отображения текста в двух разных режимах.

Внутри строки (inline) $y+2^{x+3x}$

Между строк с выравниваем по центру (display)

\[
y+2^{x+3x}
\]

Введем несколько букв из греческого алфавита, вот строчные  $\phi, \alpha, \beta, \pi$ и вот они же заглавные $\Phi, \Gamma, \Psi$

А теперь сравним разные шрифты:

$\mathrm{Mama}$: roman(upright)

$\mathit{Mama}$: italic spaced as ‘text’

$\mathbf{Mama}$: boldface

$\mathsf{Mama}$: sansserif

$\mathtt{Mama}$: monospaced(typewriter)

Попробум вложить их друг в друга:

$\mathbf{\mathrm{Mama}}$

$\mathsf{\mathbf{Mama}}$

$\mathbf{\mathit{Mama}}$

\begin{equation}
 \int_{-\infty}^{+\infty} e^{-x^2} \, dx
 \end{equation}

\end{document}