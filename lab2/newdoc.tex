%% The document class with options
 \documentclass[a4paper,12pt]{article}
 %% Select T1 font encoding: suitable for Western European Latin scripts
 \usepackage[T1]{fontenc}
\usepackage{ucs} 
\usepackage[utf8x]{inputenc} % Включаем поддержку UTF8  
\usepackage[russian]{babel}
 %% A comment in the preamble
 \begin{document}

	Это новый документ для просмотра возможностей LaTeX
	
	Здесь должен быть отдельный абзац, для наглядности я напишу сюда больше разных слов, чтобы посмотреть на то, как более длинное предложение будет выглядеть в итоговом документе.
	
	Это ёщё один абзац посмотрим как работают некоторые простые команды \{ Например предложение в фигурных скобках \} или если сейчас оценивать выполнение данной работы, то я думаю можно сказать что пока я сейчас её пишу она выполнена на 40 \%.

	А теперь посмотри как работают пробелы

	вот какой-то текст с неразрывными пробелами: А.~П.~Супонина

а если мы используем много обычных пробелов             он все равно должен быть как~1

и если мы хотим сделать более большой пробел \;\;\; делаем так 

а вот \\ перенос строки
 \end{document}